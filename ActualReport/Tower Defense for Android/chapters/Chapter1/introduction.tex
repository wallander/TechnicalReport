\chapter{Introduction}

In the beginning of 2007 a powerful development of smartphones began. Apple's introduction of the iPhone was a starting signal of this rapid development. \citep{Time} Since then, several other companies have released smartphones. The majority of them have operating systems developed by Nokia, RIM, Apple, Microsoft or Google  \citep{Canalys}.

Each smartphone operating system has its own system for publishing applications. The applications can be downloaded by users with the same operating system. It is up to the developers to charge a fee for their applications or if they want, distribute them for free. This new way of marketing applications allows everyone from bigger companies to the individual developer to compete under basically the same conditions.

The development of a game was chosen because it consists of a lot of common obstacles encountered during software development. The difficulties of programming a game include drawing graphics, managing code efficiency as well as implementing a good system for user interaction. A game is a good place to start when experimenting with new input methods, because there are no predefined rules for how a game is supposed to work.

Tower Defense is a common type of game, often found as Flash-applications on websites. The general concept is that creatures, organized into waves, are trying to get from point A to point B. The objective of the game is to prevent the creatures from reaching their destination. This is achieved by constructing towers that fire automatically at creatures that enter their range. The games often feature different towers with different purposes, as well as different creatures with varying attributes, weaknesses and strengths. 

\section{Purpose and delimitations}

The purpose of this project is to investigate how the new interaction possibilities of smartphones could be used in a real-time game environment. The goal is to develop the basis of a commercially viable game for the android platform that makes use of the touchscreen and accelerometer in new and innovative ways.

The main focus of the project is to develop a stable, extendable and correctly implemented structure, to facilitate further development. The number of tracks, tower types, mob types and different sound effects is intentionally small. Focus lies on functionality rather than quantity.

Game balance has a huge impact on commercial viability. According to the first interview an unbalanced game is perceived as less attractive to play. Due to this fact, one of the goals of this project is to make the game feel as balanced as possible. The game should be easy to play but still provide a challenge for the user.

Phone model compatibility has only been considered to a small extent. The software was only tested on three different phone models (HTC Hero, HTC Legend and HTC Desire). Differences between models in CPU power was not taken into consideration during the project.