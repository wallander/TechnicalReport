%
% This file presents the `verbose-ibid' style
%
\documentclass[a4paper]{article}
\usepackage[T1]{fontenc}
\usepackage[american]{babel}
\usepackage{csquotes}
\usepackage[style=verbose-ibid]{biblatex}
\usepackage{hyperref}
\bibliography{biblatex-examples}
\newcommand{\cmd}[1]{\texttt{\textbackslash #1}}
\begin{document}

\section*{The \texttt{verbose-ibid} style}

This citation style is a slightly more compact variant of the
\texttt{verbose} style. Immediately repeated citations are replaced
by the abbreviation `ibidem' unless the citation is the first one on
the current page or double page spread (depending on the setting of
the \texttt{pagetracker} package option). This style is also
intended for citations given in footnotes.

\subsection*{Additional package options}

\subsubsection*{The \texttt{ibidpage} option}

The scholarly abbreviation \emph{ibidem} is sometimes taken to mean
both `same author~+ same title' and `same author~+ same title~+ same
page' in traditional citation schemes. By default, this is not the
case with this style because it may lead to ambiguous citations. If
you you prefer the wider interpretation of \emph{ibidem}, set the
package option \texttt{ibidpage=true} or simply \texttt{ibidpage} in
the preamble. The default setting is \texttt{ibidpage=false}.

\subsubsection*{The \texttt{dashed} option}

By default, this style replaces recurrent authors/editors in the
bibliography by a dash so that items by the same author or editor
are visually grouped. This feature is controlled by the package
option \texttt{dashed}. Setting \texttt{dashed=false} in the
preamble will disable this feature. The default setting is
\texttt{dashed=true}.

\subsection*{Hints}

If you want terms such as \emph{ibidem} to be printed in italics,
redefine \cmd{mkibid} as follows:

\begin{verbatim}
\renewcommand*{\mkibid}{\emph}
\end{verbatim}

\subsection*{\cmd{footcite} examples}

% The initial citation of an entry includes all the data.
This is just filler text.\footcite{aristotle:anima}
This is just filler text.\footcite{aristotle:physics}
% Subsequent citations use a more compact format.
This is just filler text.\footcite{aristotle:anima}
This is just filler text.\footcite{aristotle:physics}
% Immediately repeated citations are replaced by the
% abbreviation `ibidem'...
This is just filler text.\footcite{aristotle:physics}
\clearpage
% ... unless the citation is the first one on the current page
% or double page spread (depending on the setting of the
% `pagetracker' package option).
This is just filler text.\footcite{aristotle:physics}
This is just filler text.\footcite{aristotle:physics}

\clearpage

% If the `shorthand' field is defined, the shorthand is introduced
% on the first citation.
This is just filler text.\footcite{kant:kpv}
This is just filler text.\footcite{kant:ku}
% All subsequent citations will then use the shorthand.
This is just filler text.\footcite[24]{kant:kpv}
This is just filler text.\footcite[59--63]{kant:ku}

\clearpage

\subsection*{\cmd{autocite} examples}

% The \autocite command works like \footcite. Note that
% the period is moved and placed before the footnote.

This is just filler text \autocite{aristotle:rhetoric}.
This is just filler text \autocite{averroes/bland}.
This is just filler text \autocite{aristotle:rhetoric}.
This is just filler text \autocite{aristotle:anima}.
This is just filler text \autocite{aristotle:physics}.
This is just filler text \autocite{aristotle:physics}.

\clearpage

% Since all bibliographic data is provided on the first citation,
% this style may be used without a list of references and
% shorthands. Of course these lists may still be printed if desired.

\printshorthands
\printbibliography

\end{document}
