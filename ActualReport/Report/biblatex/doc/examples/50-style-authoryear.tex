%
% This file presents the `authoryear' style
%
\documentclass[a4paper]{article}
\usepackage[T1]{fontenc}
\usepackage[american]{babel}
\usepackage{csquotes}
\usepackage[style=authoryear]{biblatex}
\usepackage{hyperref}
\bibliography{biblatex-examples}
\setlength{\parindent}{0pt}
% Some generic settings.
\newcommand{\cmd}[1]{\texttt{\textbackslash #1}}
\setlength{\parindent}{0pt}
\begin{document}

\section*{The \texttt{authoryear} style}

This style implements the author-year citation scheme.

\subsection*{Additional package options}

\subsubsection*{The \texttt{dashed} option}

By default, this style replaces recurrent authors/editors in the
bibliography by a dash so that items by the same author or editor
are visually grouped. This feature is controlled by the package
option \texttt{dashed}. Setting \texttt{dashed=false} in the
preamble will disable this feature. The default setting is
\texttt{dashed=true}.

\subsection*{\cmd{cite} examples}

\cite{companion}

\cite[59]{companion}

\cite[see][]{companion}

\cite[see][59--63]{companion}

\subsection*{\cmd{parencite} examples}

This is just filler text \parencite{companion}.

This is just filler text \parencite[59]{companion}.

This is just filler text \parencite[see][]{companion}.

This is just filler text \parencite[see][59--63]{companion}.

\subsection*{\cmd{parencite*} examples}

\citeauthor{companion} show that this is just filler
text \parencite*{companion}.

\subsection*{\cmd{footcite} examples}

% Even though the author-year scheme is usually employed in
% conjuntion with in-text citations, it works just fine in
% footnotes, too.

This is just filler text.\footcite{companion}

This is just filler text.\footcite[59]{companion}

This is just filler text.\footcite[See][]{companion}

This is just filler text.\footcite[See][59--63]{companion}

\subsection*{\cmd{textcite} examples}

\textcite{companion} show that this is just filler text.

\textcite[59]{companion} show that this is just filler text.

\textcite[see][]{companion} show that this is just filler text.

\textcite[see][59--63]{companion} show that this is just filler text.

\subsection*{\cmd{autocite} examples}

% By default, the \autocite command works like \parencite.
% The starred version works like \parencite*.

This is just filler text \autocite{companion}.

\citeauthor{companion} show that this is just filler
text \autocite*{companion}.

\subsection*{Multiple citations}

% By default, a list of multiple citations is not sorted. You can
% enable sorting by setting the `sortcites' package option.

\cite{knuth:ct:c,aristotle:physics,knuth:ct:b,aristotle:poetics,aristotle:rhetoric,knuth:ct:d}

\subsection*{Shorthand examples}

% If an entry in the bib file includes a `shorthand' field, the
% shorthand replaces the regular author-title citation.

\cite{kant:kpv,kant:ku}

\clearpage

% The list of shorthands.

\printshorthands

% The list of references. Note that the year is printed after the
% author or editor and that recurring author and editor names are
% replaced by a dash unless the entry is the first one on the
% current page or double page spread (depending on the setting of
% the `pagetracker' package option). This style will implicitly set
% `pagetracker=spread' at load time.

\nocite{*}
\printbibliography

\end{document}
