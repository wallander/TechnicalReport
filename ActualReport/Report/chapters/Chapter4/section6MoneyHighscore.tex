\section{Money and highscore}

When a mob is killed the player is rewarded with money and points. The money is used to buy more towers as the game progresses and can thus establish a stronger defense. If the player fails to kill mob, not only does he loose a life but he also misses the reward money. This makes it harder to defend against the increasingly healthier mobs.

The points are added to the player\'s total score for the current track. The score is calculated based on the starting health of the killed mobs. It is also decreased by a factor between 1-2 depending of how far the mob has travelled before it is killed. This introduces some variation to the high score, ensuring players are rewarded for a more efficient defense. With the distance included in the formula the highscore get bigger if a player kills the mobs early.

%----------
%- Code snippet change score
%----------
\begin{figure}[htb]
\begin{small}
\verbatiminput{code/changeScore.java}
\end{small}
\caption{Caption}
\label{fig:codeExChangeScore}
\end{figure}
%----------

The high score is managed by the singleton named Highscore. When first initiated, it tries to read from the file tddata.txt. If the game has started for the first time, or if no track have been completed, the file does not exist. An exception that states that no such file exists, is thrown and caught. When caught, a file is created with the name tddata.txt.

%----------
%- Code snippet read or create file
%----------
\begin{figure}[htb]
\begin{small}
\verbatiminput{code/readCreateFile.java}
\end{small}
\caption{Caption}
\label{fig:codeExReadCreateFile}
\end{figure}
%----------

Since the read and write methods are used on more than one occasion, the code for reading and writing has their own separate methods for initializing the BufferedReader and BufferedWriter, to reduce the total amount of code.

For the player to be able to progress in the game, a track has to be completed. When a track is completed the method saveCurrentTrackScore() is invoked and saves the score to the file tddata.txt. In the progression route map, the high score file is then read to determine if a certain track has been played. If a track has score stored the next track is then unlocked and becomes playable.