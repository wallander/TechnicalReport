\chapter{Theory}

\section{Android architecture}

Developing for the Android platform is done using the Java programming language. All code can be written using standard approaches of Java programming. However, the Android operating system has very large influence on how applications are executed on the system. Most operating systems on personals computers normally allow the user to run several applications concurrently in different windows, that can also be viewed simultaneously. On the Android device, there is no native way of seeing what applications are running. The hardware buttons on the device are used to either close applications or send them to the background. Since there is no feedback on what happened to the application, it is important to handle such events in a consistent manner.
 
Gaining access to the surface of an Android device requires an implementation of the Activity class. The first describing line in the Android API about activities is ''An activity is a single, focused thing that the user can do'' CITE(Android, 2010). To create an application that shows something to the user, an Activity must be implemented. If it is not important to display information to the user, the Service class may be used (a class that is designed for applications running in the background).

%-------------------
%- Image ACTIVITY DIAGRAM
%-------------------
\begin{figure}[here]
\begin{center}
\includegraphics[scale=0.5]{pics/chapters/chapter2/activity_lifecycle2}
\end{center}
\caption{Life cycle of an Activity}
\label{fig:androidActivityLifeCycle}
\end{figure}
%-------------------------------

As can be seen in figure ~\ref{fig:androidActivityLifeCycle}, onCreate(), onStart() and onResume() are all invoked as an activity is first created. Several other methods are also invoked when the operating system needs to manage memory shortage. Once the activity is up and running, it is important to handle these methods correctly. For instance, when a lot of variables are instantiated in the onStart()-method, memory leaks might occur if they are not set to null in the onStop()-method. Memory leaks can cause the entire device to slow down, which can be very frustrating for the user.
 
The graphical layout of an activity consists of objects of the View class. Views can be defined either procedurally while creating the activity, or by accessing predefined layouts from XML-files. In Android applications, views are both responsible for drawing images to the screen and for taking care of events generated from user interaction. For instance, a button is a view that can register listeners for onClick-events. Similarly, events generated by the trackball, hardware buttons and touchscreen are also handled by views.

External resources are often used when developing for Android. Any type of file can be included to the binary files when building an application. If Eclipse is being used to develop the application, a file called R.java is generated whenever the resource directory is updated. This file contains translations from integer resource pointers to variable names that are easier to understand. When the application is built, the resource files are compiled into binary files that load fast and efficiently. \\ CITE (http://developer.android.com/guide/topics/resources/resources-i18n.html)

Accessing resources is done by invoking the method getResources() on the Context that is attached to the application. Context is an interface that is implemented by fundamental Android classes, such as Activity or Service. As stated in the Android API CITE (Android 2010), "It allows access to application-specific resources and classes.".

%----------------------------------------------------------
%----------------------------------------------------------
\subsection{Data storage}

The file system on Android devices differs from systems used on personal computers. On a personal computer, file system data files for one application can be read by any other application. On Android devices however data files created by one application is only readable to that application. Android has four different solutions to store and receive data from the file system on a mobile phone; preferences, files, databases and network. CITE(Android 2010) 

The preferences solution uses key-value pairs to write simple data types, such as texts to be loaded at the start of an application, or settings the user wants to be saved for next time he starts the application. This data is a lightweight method of writing and retrieving data, and is therefor recommended to use for simple data types. CITE(Android 2010)

Another way to manage data storage is to use files. This is a basic way to handle data on the mobile phone's memory card, where files are created, written to and read from. CITE(Android 2010) 

Android also comes with the possibility of using databases for data storage. The type of database available on a Android device is SQLite. CITE(Android 2010) SQLite is a lightweight database engine, built to suit devices with limited memory. It reads and writes to files on the device\'s file system. "A complete SQL database with multiple tables, indices, triggers, and views, is contained in a single disk file." CITE(SQLite 2010)

As long as the phone is connected to the Internet, either via 3G or via a wireless network, it is possible to use the network connection to send and receive data. (Android 2010)
%----------------------------------------------------------
%----------------------------------------------------------
\subsection{Graphics}

There are three approaches to handling graphics when working with Android. The first approach uses predefined layouts in XML-files. The other two approaches involve the Java class Canvas or the cross-language standard specification OpenGL. The Canvas class provides simple tools like rectangles, color filters and bitmaps that let you draw pictures on the screen. It is handled like layers even if it is not exactly layers; the last object that is drawn on the canvas will be drawn on top of anything that was drawn earlier. Canvas only supports two axes, x- and y-coordinates compared to OpenGL that has full support for programming 3D graphics.

For each activity using a predefined layout, there has to be an XML file describing the layout. There is a built-in XML-editor in the Android Software Development Kit for Eclipse that you can use to create these layouts. The editor shows a preview of the window that will be shown on the screen of the device (see figure ~\ref{fig:xmlEditor}). Inside that window is where all the graphics are put. The editor is very easy to use as it uses the drag-and-drop concept. There are several layout options such as GridView, ListView and LinearLayout to choose between and combine. There are also several view options such as normal View, Button, Checkbox, TextView and many more. Items are dragged and dropped to their correct positions. There is also a property window for every item, that gives access to customizing that particular item in the layout. 

%-------------------------
%- IMAGE XML Editor
%-------------------------
\begin{figure}[here]
\begin{center}
\includegraphics[scale=0.3]{pics/chapters/chapter2/xmleditor}
\end{center}

\caption{Caption for the xmleditor}
\label{fig:xmlEditor}

\end{figure}
%-------------------------

%----------------------------------------------------------
%----------------------------------------------------------
\subsection{Sound}

The Android operating system is able to provide playback of different media files. A list of all supported media formats is published in the Android API (Android 2010). When considering sound files in particular, it is clear that all of the major sound formats are supported (MP3, MIDI, WAVE, Ogg Vorbis and more). Unless playing sounds is the main purpose of the application, it is desirable to use files that are small and do not require lots of memory. 

Implementations of using sound in Android applications are done by utilizing the classes MediaPlayer and SoundPool. The SoundPool class should be used for small sounds, sounds that are likely to be repeated a lot. Properties of this class are described in the Android API (Android 2010). Sounds added to a SoundPool are decoded by a MediaPlayer and stored as raw bitstreams. Not having to decode sound files every time they are played reduces the risk of experiencing performance issues due to heavy CPU load. MediaPlayer is better used for long sound files, such as background music in games. 
%----------------------------------------------------------