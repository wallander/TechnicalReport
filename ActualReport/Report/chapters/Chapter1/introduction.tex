\chapter{Introduction}

\section{Background}

In the beginning of 2007 a powerful development of smartphones in the mobile phone market was introduced. The introduction of the Iphone by Apple was the starting signal of this rapid development. Since then, several other IT companies have released this type of phone. Most of these smartphones have operating systems from Symbian, RIM, Apple, Microsoft or Google CITE"(Canalys, 2010)". What they all have in common is that it is relatively simple to develop software for them, meaning that anyone with basic programming skills can develop their own applications. 

Every smartphone operating system has their own system for publishing applications. These applications can be downloaded from users with the same operating system. It is up to the developers to charge a fee for their applications or if they want, give them away for free. This new way of marketing applications allows everyone from bigger companies to the individual developer to compete under the same conditions.

The development of a game was chosen because it consists of a lot of the common obstacles encountered during software development. The difficulties of programming a game include drawing of graphics, managing code efficiency as well as implementing a good system for user interaction. A game is also a good place to start when experimenting with new input methods. The concept of Tower Defense was chosen for the fact that there were few if any Tower Defense games that featured the use of an accelerometer in its game design. 

A Tower Defense is a common type of game often found as Flash-applications on websites. The general concept is that the games feature creatures, organized into waves, that try to go from point A to point B. The objective of the game is to prevent the creatures from reaching their final destination. This is achieved by constructing towers that fire automatically and intermittently upon creatures that enter their range. The game is often varied by introducing different paths creatures can take; some games allow players to construct obstacles that force the mobs to take a certain path, giving the towers additional time to fire. The games often feature different towers with different purposes, as well as different creatures with varying attributes, weaknesses and strengths. 

\section{Purpose and delimitations}

The purpose of this project is to investigate how the new interaction possibilities of smartphones could be used in a real-time game environment. The goal is to develop the basis of a commercially viable game for the android platform that makes use of the touchscreen and accelerometer in new and innovative ways.

The main focus of the project is to develop a stable, extendable and correctly implemented structure for the game, so that further development is facilitated. The number of tracks, tower types, mob types and different sound effects is intentionally small, since focus lies on functionality rather than quantity.
Game balance has a huge impact on commercial viability. If people think the game is unbalanced, they might not want to play it. Due to this fact, one of the goals of this project is to make the game feel as balanced as possible. This can be done by altering values and properties of different game objects, creating synergy and making all elements of the game attractive to the user. The game should be easy to play but still provide a challenge for the user.

Phone model compatibility has not been a main focus. Since the Android platform is designed to provide developers and phone manufacturers with a good foundation on which to base their software, the platform inherently has a good degree of cross-model robustness. However, in order to ensure proper compatibility, the software was tested on three different phone models (HTC Hero, HTC Legend and HTC Desire). Another difference between models is the CPU power and this fact was not taken into consideration during the project. More information on this can be found under ''future work''.