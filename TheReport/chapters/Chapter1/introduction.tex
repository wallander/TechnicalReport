\chapter{Introduction}

\section{Background}

The last year saw a powerful development of so called Smart-Phones in the western world. The introduction of iPhone by Apple Inc. was the starting signal of this rapid development. Since then, several other developers have released this type of phone. These smartphones (v�lj en stavning!) have operating systems from Apple, Windows or Google. What they all have in common is that it is relatively simple to develop software for them and anyone with basic programming skills can introduce their software onto the phone.

Coupled with this is an expansive system for marketing these applications, where users themselves can make their work available to others. The applications can be sold for money or given away for free. These new phones also make extensive use of alternative input methods such as touch-sensitive screens, expanded voice commands and an accelerometer which detects which way the phone is moved or tilted.

% Lite smidigare �verg�ng h�r

A Tower Defense is a common type of game often found as Flash-applications on websites. They all differ somewhat, but what they all have in common is that they feature creatures, organized into waves that try to go from point A to point B. The objective of the game is to prevent this from happening, which is achieved by constructing towers that fire automatically and intermittently upon creatures that enter their range. The game is often varied by introducing different paths these creatures can take, some games even allowing the players themselves to construct obstacles that force the creatures to take a certain path which allows your towers additional time to fire. The game often features different towers with different purposes, as well as different creatures with varying attributes, weaknesses and strengths. 

The development of a game was chosen because it represents a lot of the common obstacles encountered when developing software. It provides you with the challenges of drawing on a canvas, managing the efficiency of the code to not slow down the device as well as implementing a good system for user interaction. A game is also a good place to start out when experimenting with new input methods. The concept of tower defense was chosen for the fact that there were few if any tower defense games that featured the use of an accelerometer in its game design.

\section{Purpose and delimitations}

The purpose of this project is to investigate how the new interaction possibilities of smart-phones could be used for a real-time game environment. The goal is to develop the basis for a commersially viable game for the android platform that makes use of the touch screen and accelerometer in new and innovative ways.

The main focus of the project is to develop a stable, extendible and correctly implemented structure for the game, so that further development is facilitated. The number of tracks, tower types, mob types and different sound effects is intentionally small, since focus lies on functionality rather than quantity.

Phone model compability has not been a main focus. Since the Android platform is designed to provide developers and phone manufacturers with an good foundation on which to base their software, the platform inherenltly has a good degree of cross-model robustness. However, in order to ensure proper compatibility, we tested the software on three different phone models (HTC Hero, HTC Legendand HTC Desire), and made some necessary adjustments to accomodate to different screen sizes. Another difference between models is the CPU power, and this fact was not taken into consideration during the project. More information on this can be found under "future work".